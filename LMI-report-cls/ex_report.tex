\DocumentMetadata{
    pdfversion=1.7,
    pdfstandard=A-3b,
}
\documentclass[11pt,a4paper]{LMIReport}

\Metrologist{Luke Skywalker}
\ChiefMetrologist{Princess Leia Organa}
\ReportNumber{12345} 
\ReportTitle{A report on the calibration of an type-N male open}
\date{17 May 2035}

\begin{document}	
\section{Description}
The components are from a USC vector network analyser calibration kit model 8599. 

\section{Identification}
The component serial number is 2221X.

\section{Client}
United Spacecraft Corporation, 51 Mare Tranquillitatis, The Moon.

\section{Date of Calibration}
The measurements were performed on February 7\textsuperscript{th}, 2035.

\section{Conditions}
Ambient temperature was maintained within \SI{\pm 1}{\celsius} of \SI{-123}{\celsius}.

\section{Method}
Measurements of the voltage reflection coefficient were made according to procedure LMIT.E.063.005. 

\clearpage    % Anticipate the page break
\section{Results}
 

% Although not used here, it is also possible to have a \subsubsection{}
\subsection{Open (male), SN 54673}

\begin{center} 
\begin{singlespace}
 	\small	% smaller font size for the table entries
 
  	% Increases the vertical spacing between rows slightly  
  	\setlength{\extrarowheight}{3pt}
  
	\[
		% the 'S' array column type will align numbers on the decimal 
        % Note 'S[group-minimum-digits=3]' or '\sisetup{group-minimum-digits=3 }'
        % would be used to force a space separator every 3 digits (this
        % does not happen by default until there more than 4 digits)
  		\begin{array}{SSSSS}
    		\multicolumn{1}{c}{ \text{frequency} } & 
    		\multicolumn{2}{c}{ \text{magnitude} } &
    		\multicolumn{2}{c}{ \text{phase} } 
    		\\
		% 2nd line 
    		\multicolumn{1}{c}{ (/\si{\mega\hertz}) } &  
    		\multicolumn{2}{c}{  } &
    		\multicolumn{2}{c}{ (/\si{\degree}) } 
    		\\
  		% 3rd line 
     		& {\rho} & {U(\rho)} & {\phi} & {U(\phi)} 
     		\\ \hline % Underline the headings

  		%%-----------------------------------------------
  		% Data here
		45 &   0.9998 &   0.0023$^\dagger$ &    -1.46 &     0.13     \\
		50 &   0.9998 &   0.0023$^\dagger$ &    -1.62 &     0.13     \\
		100 &   0.9999 &   0.0023$^\dagger$ &    -3.27 &     0.13    \\
		300 &   0.9998 &   0.0025 &    -9.80 &     0.14    \\
		500 &   0.9997 &   0.0026 &   -16.34 &     0.15    \\
		1000 &   1.0000 &   0.0032 &   -32.72 &     0.18   \\
		2000 &   0.9994 &   0.0054 &   -65.67 &     0.31  \\
		3000 &    1.000 &    0.011 &   -98.66 &     0.62   \\
		4000 &    0.999 &    0.013 &  -131.74 &     0.78   \\
		5000 &    0.999 &    0.016 &  -164.77 &     0.90   \\
		6000 &    0.998 &    0.017 &  +162.15 &     0.99   \\
		7000 &    0.997 &    0.018 &   +129.0 &      1.1   \\
		8000 &    0.997 &    0.018 &    +95.9 &      1.1   \\
		9000 &    0.996 &    0.018 &    +62.7 &      1.1  \\
		%%-----------------------------------------------
		
		\end{array}
	\]
	\LMICaption{table}{tab1}{%
Magnitude and phase data, using a linear scale for magnitude and units of degrees for phase. 
Expanded uncertainties decorated by a $\dagger$ fall outside the scope of accreditation (see Uncertainty section).
}
	
\end{singlespace}
\end{center}

\section{Uncertainty}
A coverage factor $k=1.96$ was used to calculate the expanded uncertainties $U(\cdot)$ at a level of confidence of approximately \SI{95}{\percent}. 
The number of degrees of freedom associated with each measurement result was large enough to justify this coverage factor.  

Some of the expanded uncertainty values reported fall outside LMI's current scope of accreditation. 
These values are decorated by a $\dagger$ in Table~\ref{tab1}. 
The least expanded uncertainty for a measured magnitude close to unity in the LMI scope of accreditation is currently 0.0024. 

\paragraph{Note:} \referenceGUM	% Standard reference to the GUM

\embedfile[mimetype=application/vnd.openxmlformats-officedocument.spreadsheetml.sheet]{ex_data.xlsx}

\end{document}